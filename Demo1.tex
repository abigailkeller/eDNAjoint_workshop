% Options for packages loaded elsewhere
\PassOptionsToPackage{unicode}{hyperref}
\PassOptionsToPackage{hyphens}{url}
\documentclass[
]{article}
\usepackage{xcolor}
\usepackage[margin=1in]{geometry}
\usepackage{amsmath,amssymb}
\setcounter{secnumdepth}{-\maxdimen} % remove section numbering
\usepackage{iftex}
\ifPDFTeX
  \usepackage[T1]{fontenc}
  \usepackage[utf8]{inputenc}
  \usepackage{textcomp} % provide euro and other symbols
\else % if luatex or xetex
  \usepackage{unicode-math} % this also loads fontspec
  \defaultfontfeatures{Scale=MatchLowercase}
  \defaultfontfeatures[\rmfamily]{Ligatures=TeX,Scale=1}
\fi
\usepackage{lmodern}
\ifPDFTeX\else
  % xetex/luatex font selection
\fi
% Use upquote if available, for straight quotes in verbatim environments
\IfFileExists{upquote.sty}{\usepackage{upquote}}{}
\IfFileExists{microtype.sty}{% use microtype if available
  \usepackage[]{microtype}
  \UseMicrotypeSet[protrusion]{basicmath} % disable protrusion for tt fonts
}{}
\makeatletter
\@ifundefined{KOMAClassName}{% if non-KOMA class
  \IfFileExists{parskip.sty}{%
    \usepackage{parskip}
  }{% else
    \setlength{\parindent}{0pt}
    \setlength{\parskip}{6pt plus 2pt minus 1pt}}
}{% if KOMA class
  \KOMAoptions{parskip=half}}
\makeatother
\usepackage{color}
\usepackage{fancyvrb}
\newcommand{\VerbBar}{|}
\newcommand{\VERB}{\Verb[commandchars=\\\{\}]}
\DefineVerbatimEnvironment{Highlighting}{Verbatim}{commandchars=\\\{\}}
% Add ',fontsize=\small' for more characters per line
\usepackage{framed}
\definecolor{shadecolor}{RGB}{248,248,248}
\newenvironment{Shaded}{\begin{snugshade}}{\end{snugshade}}
\newcommand{\AlertTok}[1]{\textcolor[rgb]{0.94,0.16,0.16}{#1}}
\newcommand{\AnnotationTok}[1]{\textcolor[rgb]{0.56,0.35,0.01}{\textbf{\textit{#1}}}}
\newcommand{\AttributeTok}[1]{\textcolor[rgb]{0.13,0.29,0.53}{#1}}
\newcommand{\BaseNTok}[1]{\textcolor[rgb]{0.00,0.00,0.81}{#1}}
\newcommand{\BuiltInTok}[1]{#1}
\newcommand{\CharTok}[1]{\textcolor[rgb]{0.31,0.60,0.02}{#1}}
\newcommand{\CommentTok}[1]{\textcolor[rgb]{0.56,0.35,0.01}{\textit{#1}}}
\newcommand{\CommentVarTok}[1]{\textcolor[rgb]{0.56,0.35,0.01}{\textbf{\textit{#1}}}}
\newcommand{\ConstantTok}[1]{\textcolor[rgb]{0.56,0.35,0.01}{#1}}
\newcommand{\ControlFlowTok}[1]{\textcolor[rgb]{0.13,0.29,0.53}{\textbf{#1}}}
\newcommand{\DataTypeTok}[1]{\textcolor[rgb]{0.13,0.29,0.53}{#1}}
\newcommand{\DecValTok}[1]{\textcolor[rgb]{0.00,0.00,0.81}{#1}}
\newcommand{\DocumentationTok}[1]{\textcolor[rgb]{0.56,0.35,0.01}{\textbf{\textit{#1}}}}
\newcommand{\ErrorTok}[1]{\textcolor[rgb]{0.64,0.00,0.00}{\textbf{#1}}}
\newcommand{\ExtensionTok}[1]{#1}
\newcommand{\FloatTok}[1]{\textcolor[rgb]{0.00,0.00,0.81}{#1}}
\newcommand{\FunctionTok}[1]{\textcolor[rgb]{0.13,0.29,0.53}{\textbf{#1}}}
\newcommand{\ImportTok}[1]{#1}
\newcommand{\InformationTok}[1]{\textcolor[rgb]{0.56,0.35,0.01}{\textbf{\textit{#1}}}}
\newcommand{\KeywordTok}[1]{\textcolor[rgb]{0.13,0.29,0.53}{\textbf{#1}}}
\newcommand{\NormalTok}[1]{#1}
\newcommand{\OperatorTok}[1]{\textcolor[rgb]{0.81,0.36,0.00}{\textbf{#1}}}
\newcommand{\OtherTok}[1]{\textcolor[rgb]{0.56,0.35,0.01}{#1}}
\newcommand{\PreprocessorTok}[1]{\textcolor[rgb]{0.56,0.35,0.01}{\textit{#1}}}
\newcommand{\RegionMarkerTok}[1]{#1}
\newcommand{\SpecialCharTok}[1]{\textcolor[rgb]{0.81,0.36,0.00}{\textbf{#1}}}
\newcommand{\SpecialStringTok}[1]{\textcolor[rgb]{0.31,0.60,0.02}{#1}}
\newcommand{\StringTok}[1]{\textcolor[rgb]{0.31,0.60,0.02}{#1}}
\newcommand{\VariableTok}[1]{\textcolor[rgb]{0.00,0.00,0.00}{#1}}
\newcommand{\VerbatimStringTok}[1]{\textcolor[rgb]{0.31,0.60,0.02}{#1}}
\newcommand{\WarningTok}[1]{\textcolor[rgb]{0.56,0.35,0.01}{\textbf{\textit{#1}}}}
\usepackage{graphicx}
\makeatletter
\newsavebox\pandoc@box
\newcommand*\pandocbounded[1]{% scales image to fit in text height/width
  \sbox\pandoc@box{#1}%
  \Gscale@div\@tempa{\textheight}{\dimexpr\ht\pandoc@box+\dp\pandoc@box\relax}%
  \Gscale@div\@tempb{\linewidth}{\wd\pandoc@box}%
  \ifdim\@tempb\p@<\@tempa\p@\let\@tempa\@tempb\fi% select the smaller of both
  \ifdim\@tempa\p@<\p@\scalebox{\@tempa}{\usebox\pandoc@box}%
  \else\usebox{\pandoc@box}%
  \fi%
}
% Set default figure placement to htbp
\def\fps@figure{htbp}
\makeatother
\setlength{\emergencystretch}{3em} % prevent overfull lines
\providecommand{\tightlist}{%
  \setlength{\itemsep}{0pt}\setlength{\parskip}{0pt}}
\usepackage{bookmark}
\IfFileExists{xurl.sty}{\usepackage{xurl}}{} % add URL line breaks if available
\urlstyle{same}
\hypersetup{
  pdftitle={Demo 1: Endangered tidewater gobies},
  hidelinks,
  pdfcreator={LaTeX via pandoc}}

\title{Demo 1: Endangered tidewater gobies}
\author{}
\date{\vspace{-2.5em}}

\begin{document}
\maketitle

First we will load the package:

\begin{Shaded}
\begin{Highlighting}[]
\CommentTok{\# load the package}
\FunctionTok{library}\NormalTok{(eDNAjoint)}
\FunctionTok{library}\NormalTok{(tidyverse)}
\end{Highlighting}
\end{Shaded}

\begin{verbatim}
## -- Attaching core tidyverse packages ------------------------ tidyverse 2.0.0 --
## v dplyr     1.1.4     v readr     2.1.5
## v forcats   1.0.0     v stringr   1.5.1
## v ggplot2   3.5.1     v tibble    3.2.1
## v lubridate 1.9.4     v tidyr     1.3.1
## v purrr     1.0.2     
## -- Conflicts ------------------------------------------ tidyverse_conflicts() --
## x dplyr::filter() masks stats::filter()
## x dplyr::lag()    masks stats::lag()
## i Use the conflicted package (<http://conflicted.r-lib.org/>) to force all conflicts to become errors
\end{verbatim}

\begin{Shaded}
\begin{Highlighting}[]
\FunctionTok{library}\NormalTok{(bayestestR)}
\FunctionTok{library}\NormalTok{(patchwork)}
\end{Highlighting}
\end{Shaded}

\section{Prepare the data}\label{prepare-the-data}

Ensuring that your data is formatted correctly is essential for
successfully using eDNAjoint. Let's first explore the structure of the
goby data:

\begin{Shaded}
\begin{Highlighting}[]
\FunctionTok{data}\NormalTok{(goby\_data)}
\FunctionTok{str}\NormalTok{(goby\_data)}
\end{Highlighting}
\end{Shaded}

\begin{verbatim}
## List of 4
##  $ pcr_n   : num [1:39, 1:22] 6 6 6 6 6 6 6 6 6 6 ...
##   ..- attr(*, "dimnames")=List of 2
##   .. ..$ : NULL
##   .. ..$ : chr [1:22] "1" "2" "3" "4" ...
##  $ pcr_k   : num [1:39, 1:22] 0 0 0 0 6 0 0 0 0 5 ...
##   ..- attr(*, "dimnames")=List of 2
##   .. ..$ : NULL
##   .. ..$ : chr [1:22] "1" "2" "3" "4" ...
##  $ count   : int [1:39, 1:22] 0 0 0 0 0 0 0 0 0 1 ...
##   ..- attr(*, "dimnames")=List of 2
##   .. ..$ : NULL
##   .. ..$ : chr [1:22] "1" "2" "3" "4" ...
##  $ site_cov: num [1:39, 1:5] -0.711 -0.211 -1.16 -0.556 -0.988 ...
##   ..- attr(*, "dimnames")=List of 2
##   .. ..$ : NULL
##   .. ..$ : chr [1:5] "Salinity" "Filter_time" "Other_fishes" "Hab_size" ...
\end{verbatim}

The input data is a list of matrices, where the rows in all matrices
correspond to the number of sites.

\begin{Shaded}
\begin{Highlighting}[]
\FunctionTok{names}\NormalTok{(goby\_data)}
\end{Highlighting}
\end{Shaded}

\begin{verbatim}
## [1] "pcr_n"    "pcr_k"    "count"    "site_cov"
\end{verbatim}

\subsection{Count data}\label{count-data}

Let's look at the dimensions of the count data:

\begin{Shaded}
\begin{Highlighting}[]
\FunctionTok{dim}\NormalTok{(goby\_data}\SpecialCharTok{$}\NormalTok{count)}
\end{Highlighting}
\end{Shaded}

\begin{verbatim}
## [1] 39 22
\end{verbatim}

Number of primary samples (sites) = 39 Maximum number of secondary
samples (replicates per site) = 22

These are the number of gobies in each seine sample, at each site.

\begin{Shaded}
\begin{Highlighting}[]
\FunctionTok{head}\NormalTok{(goby\_data}\SpecialCharTok{$}\NormalTok{count)}
\end{Highlighting}
\end{Shaded}

\begin{verbatim}
##      1 2 3  4  5  6  7   8  9 10 11 12 13 14 15 16 17 18 19 20 21 22
## [1,] 0 0 0  0  0  0  0   0  0  0  0 NA NA NA NA NA NA NA NA NA NA NA
## [2,] 0 0 0  0  0  0  0   0  0  0  0 NA NA NA NA NA NA NA NA NA NA NA
## [3,] 0 0 0  0  0  0  0   0  0  0  0 NA NA NA NA NA NA NA NA NA NA NA
## [4,] 0 4 1  0  2  1 38 112  1 15 NA NA NA NA NA NA NA NA NA NA NA NA
## [5,] 0 0 0  2  0  0  0   0  0  0  0  0  4  1  0  2  0  8 NA NA NA NA
## [6,] 0 0 0 NA NA NA NA  NA NA NA NA NA NA NA NA NA NA NA NA NA NA NA
\end{verbatim}

Because we are building these matrices based on the maximum dimensions,
we fill in the matrix with NAs for samples that don't exist. For
example, at site 1, there were only 11 seine replicates, so columns
12-22 in row 1 are NA.

Site 11 has the maximum number of replicate observations (22):

\begin{Shaded}
\begin{Highlighting}[]
\NormalTok{goby\_data}\SpecialCharTok{$}\NormalTok{count[}\DecValTok{11}\NormalTok{,]}
\end{Highlighting}
\end{Shaded}

\begin{verbatim}
##   1   2   3   4   5   6   7   8   9  10  11  12  13  14  15  16  17  18  19  20 
##  58  44  54   7  27  49  56   7  27  15   0   0   4   2   5 217   0   0   0   0 
##  21  22 
##   0  14
\end{verbatim}

\subsection{eDNA (PCR) data}\label{edna-pcr-data}

Next let's look at the PCR data. \texttt{pcr\_n} is the number of PCR
replicates, per eDNA secondary sample (water sample), per site. These
are the PCR ``attempts''.

\begin{Shaded}
\begin{Highlighting}[]
\FunctionTok{head}\NormalTok{(goby\_data}\SpecialCharTok{$}\NormalTok{pcr\_n)}
\end{Highlighting}
\end{Shaded}

\begin{verbatim}
##      1 2 3  4  5  6  7  8  9 10 11 12 13 14 15 16 17 18 19 20 21 22
## [1,] 6 6 6  6  6  6  6  6  6  6  6 NA NA NA NA NA NA NA NA NA NA NA
## [2,] 6 6 6  6  6  6  6  6  6  6  6 NA NA NA NA NA NA NA NA NA NA NA
## [3,] 6 6 6  6  6  6  6  6  6  6  6 NA NA NA NA NA NA NA NA NA NA NA
## [4,] 6 6 6  6  6  6  6  6  6  6 NA NA NA NA NA NA NA NA NA NA NA NA
## [5,] 6 6 6  6  6  6  6  6  6  6  6  6  6  6  6  6  6  6  6  6 NA NA
## [6,] 6 6 6 NA NA NA NA NA NA NA NA NA NA NA NA NA NA NA NA NA NA NA
\end{verbatim}

How many PCR replicates were there at site 4, water sample 6?

\begin{Shaded}
\begin{Highlighting}[]
\FunctionTok{as.numeric}\NormalTok{(goby\_data}\SpecialCharTok{$}\NormalTok{pcr\_n[}\DecValTok{4}\NormalTok{, }\DecValTok{6}\NormalTok{])}
\end{Highlighting}
\end{Shaded}

\begin{verbatim}
## [1] 6
\end{verbatim}

Again, site 11 had the maximum number of water samples:

\begin{Shaded}
\begin{Highlighting}[]
\NormalTok{goby\_data}\SpecialCharTok{$}\NormalTok{pcr\_n[}\DecValTok{11}\NormalTok{,]}
\end{Highlighting}
\end{Shaded}

\begin{verbatim}
##  1  2  3  4  5  6  7  8  9 10 11 12 13 14 15 16 17 18 19 20 21 22 
##  6  6  6  6  6  6  6  6  6  6  6  6  6  6  6  6  6  6  6  6  6  6
\end{verbatim}

\texttt{pcr\_k} is the number of PCR successes. Of the number of PCR
replicates, how many were detections were there?

\begin{Shaded}
\begin{Highlighting}[]
\FunctionTok{head}\NormalTok{(goby\_data}\SpecialCharTok{$}\NormalTok{pcr\_k)}
\end{Highlighting}
\end{Shaded}

\begin{verbatim}
##      1 2 3  4  5  6  7  8  9 10 11 12 13 14 15 16 17 18 19 20 21 22
## [1,] 0 0 0  0  0  0  0  0  0  0  0 NA NA NA NA NA NA NA NA NA NA NA
## [2,] 0 0 0  0  0  0  0  0  0  0  0 NA NA NA NA NA NA NA NA NA NA NA
## [3,] 0 0 0  0  0  0  0  0  0  0  0 NA NA NA NA NA NA NA NA NA NA NA
## [4,] 0 6 6  4  6  5  4  6  5  3 NA NA NA NA NA NA NA NA NA NA NA NA
## [5,] 6 6 4  6  6  6  5  4  2  2  0  6  5  5  6  6  6  5  5  4 NA NA
## [6,] 0 0 0 NA NA NA NA NA NA NA NA NA NA NA NA NA NA NA NA NA NA NA
\end{verbatim}

How many PCR detections (successes) were there at site 4, water sample
6?

\begin{Shaded}
\begin{Highlighting}[]
\FunctionTok{as.numeric}\NormalTok{(goby\_data}\SpecialCharTok{$}\NormalTok{pcr\_k[}\DecValTok{4}\NormalTok{, }\DecValTok{6}\NormalTok{])}
\end{Highlighting}
\end{Shaded}

\begin{verbatim}
## [1] 5
\end{verbatim}

A few checks to make sure the data is structured correctly:

The locations of the NA should be the same in the \texttt{pcr\_k} and
\texttt{pcr\_n} matrices.

\begin{Shaded}
\begin{Highlighting}[]
\FunctionTok{all}\NormalTok{(}\FunctionTok{which}\NormalTok{(}\FunctionTok{is.na}\NormalTok{(goby\_data}\SpecialCharTok{$}\NormalTok{pcr\_k)) }\SpecialCharTok{==} \FunctionTok{which}\NormalTok{(}\FunctionTok{is.na}\NormalTok{(goby\_data}\SpecialCharTok{$}\NormalTok{pcr\_n)))}
\end{Highlighting}
\end{Shaded}

\begin{verbatim}
## [1] TRUE
\end{verbatim}

Both PCR and count data should have the same number of sites (i.e.,
number of rows):

\begin{Shaded}
\begin{Highlighting}[]
\FunctionTok{all.equal}\NormalTok{(}\FunctionTok{nrow}\NormalTok{(goby\_data}\SpecialCharTok{$}\NormalTok{pcr\_k), }\FunctionTok{nrow}\NormalTok{(goby\_data}\SpecialCharTok{$}\NormalTok{pcr\_n), }\FunctionTok{nrow}\NormalTok{(goby\_data}\SpecialCharTok{$}\NormalTok{count))}
\end{Highlighting}
\end{Shaded}

\begin{verbatim}
## [1] TRUE
\end{verbatim}

\subsection{Site-level covariate data}\label{site-level-covariate-data}

Site-level covariate data is totally optional! PCR and count data are
the minimum.

\begin{Shaded}
\begin{Highlighting}[]
\FunctionTok{head}\NormalTok{(goby\_data}\SpecialCharTok{$}\NormalTok{site\_cov)}
\end{Highlighting}
\end{Shaded}

\begin{verbatim}
##        Salinity Filter_time Other_fishes   Hab_size Veg
## [1,] -0.7114925       -1.17   -0.4738419 -0.2715560   0
## [2,] -0.2109183       -1.24   -0.4738419 -0.2663009   0
## [3,] -1.1602831       -1.29   -0.4738419 -0.2717707   0
## [4,] -0.5561419        0.11    0.5479118 -0.2164312   1
## [5,] -0.9876713       -0.70    0.2437353  4.9981956   1
## [6,]  1.2562818       -0.55   -0.3512823 -0.2934710   0
\end{verbatim}

Notice that the continuous data is normalized:

\begin{Shaded}
\begin{Highlighting}[]
\FunctionTok{mean}\NormalTok{(goby\_data}\SpecialCharTok{$}\NormalTok{site\_cov[, }\StringTok{"Salinity"}\NormalTok{])}
\end{Highlighting}
\end{Shaded}

\begin{verbatim}
## [1] -3.974359e-09
\end{verbatim}

\begin{Shaded}
\begin{Highlighting}[]
\FunctionTok{sd}\NormalTok{(goby\_data}\SpecialCharTok{$}\NormalTok{site\_cov[, }\StringTok{"Salinity"}\NormalTok{])}
\end{Highlighting}
\end{Shaded}

\begin{verbatim}
## [1] 1
\end{verbatim}

\section{Fit the model}\label{fit-the-model}

Now that we understand our data, let's fit the model using the function
\texttt{joint\_model}. The key arguments of this function include:

\begin{itemize}
\item
  \texttt{data}: list of pcr\_k, pcr\_n, count, and site\_cov matrices
\item
  \texttt{cov}: character vector of site-level covariates
\item
  \texttt{family}: probability distribution used to model the seine
  count data. Options include a poisson, negative binomial, and gamma
\item
  \texttt{p10\_priors}: Beta distribution parameters for the prior on
  the probability of false positive eDNA detection, p10. c(1,20) is the
  default specification. More on this later.
\item
  \texttt{q}: logical value indicating the presence of multiple
  traditional gear types.
\end{itemize}

\begin{Shaded}
\begin{Highlighting}[]
\CommentTok{\# run the joint model with two covariates}
\NormalTok{goby\_fit\_cov1 }\OtherTok{\textless{}{-}} \FunctionTok{joint\_model}\NormalTok{(}
  \AttributeTok{data =}\NormalTok{ goby\_data, }\CommentTok{\# data}
  \AttributeTok{cov =} \FunctionTok{c}\NormalTok{(}\StringTok{"Filter\_time"}\NormalTok{, }\StringTok{"Salinity"}\NormalTok{), }\CommentTok{\# site{-}level covariates}
  \AttributeTok{family =} \StringTok{"poisson"}\NormalTok{, }\CommentTok{\# distribution for traditional data}
  \AttributeTok{p10\_priors =} \FunctionTok{c}\NormalTok{(}\DecValTok{1}\NormalTok{, }\DecValTok{20}\NormalTok{), }\CommentTok{\# specify prior distribution for p10}
  \AttributeTok{q =} \ConstantTok{FALSE}\NormalTok{, }\CommentTok{\# only one traditional gear type}
  \AttributeTok{multicore =} \ConstantTok{TRUE} \CommentTok{\# run MCMC chains in parallel}
\NormalTok{)}
\end{Highlighting}
\end{Shaded}

There are many more arguments, including more to customize the MCMC
algorithm, but these are the primary arguments.

Now let's look at the return object:

\begin{Shaded}
\begin{Highlighting}[]
\FunctionTok{names}\NormalTok{(goby\_fit\_cov1)}
\end{Highlighting}
\end{Shaded}

\begin{verbatim}
## [1] "model" "inits"
\end{verbatim}

The first element of this list is the model, of class \texttt{stanfit}.
This model object can be used with functions in the \texttt{rstan}
package.

\begin{Shaded}
\begin{Highlighting}[]
\FunctionTok{class}\NormalTok{(goby\_fit\_cov1}\SpecialCharTok{$}\NormalTok{model)}
\end{Highlighting}
\end{Shaded}

\begin{verbatim}
## [1] "stanfit"
## attr(,"package")
## [1] "rstan"
\end{verbatim}

And the second element of the list are the initial values used to start
the MCMC. And this will indicate the initial values used for each of the
four MCMC chains.

\begin{Shaded}
\begin{Highlighting}[]
\FunctionTok{length}\NormalTok{(goby\_fit\_cov1}\SpecialCharTok{$}\NormalTok{inits)}
\end{Highlighting}
\end{Shaded}

\begin{verbatim}
## [1] 4
\end{verbatim}

\section{Model selection}\label{model-selection}

In the first model, we used salinity and filter time as our site level
covariates. Perhaps we want to see how this model compares to models
that include other covariates.

Here we will include the binary variable that indicates vegetation
presence:

\begin{Shaded}
\begin{Highlighting}[]
\CommentTok{\# fit a new model with one site{-}level covariate}
\NormalTok{goby\_fit\_cov2 }\OtherTok{\textless{}{-}} \FunctionTok{joint\_model}\NormalTok{(}\AttributeTok{data =}\NormalTok{ goby\_data, }
                             \AttributeTok{cov =} \StringTok{"Veg"}\NormalTok{,}
                             \AttributeTok{family =} \StringTok{"poisson"}\NormalTok{, }
                             \AttributeTok{p10\_priors =} \FunctionTok{c}\NormalTok{(}\DecValTok{1}\NormalTok{, }\DecValTok{20}\NormalTok{),}
                             \AttributeTok{q =} \ConstantTok{FALSE}\NormalTok{, }
                             \AttributeTok{multicore =} \ConstantTok{TRUE}\NormalTok{,}
                             \AttributeTok{verbose =} \ConstantTok{FALSE} \CommentTok{\# don\textquotesingle{}t print messages!}
\NormalTok{                            )}
\end{Highlighting}
\end{Shaded}

\begin{verbatim}
## Refer to the eDNAjoint guide for visualization tips:  https://ednajoint.netlify.app/tips#visualization-tips
\end{verbatim}

And now we will use the \texttt{joint\_select} function to compare our
two models. We will input our two models as a list:

\begin{Shaded}
\begin{Highlighting}[]
\CommentTok{\# perform model selection}
\FunctionTok{joint\_select}\NormalTok{(}\AttributeTok{model\_fits =} \FunctionTok{list}\NormalTok{(goby\_fit\_cov1}\SpecialCharTok{$}\NormalTok{model, goby\_fit\_cov2}\SpecialCharTok{$}\NormalTok{model))}
\end{Highlighting}
\end{Shaded}

\begin{verbatim}
##        elpd_diff se_diff
## model1   0.0       0.0  
## model2 -32.6      30.1
\end{verbatim}

This is performing model selection with leave-one-out cross validation
and is a measure of how well a model predicts new, unseen data. A higher
ELPD (Expected Log Predictive Density) value indicates better predictive
performance, meaning the model is better at predicting new data.

You could test this with all the covariates, but we will stick with
model 1 (salinity and filter time) for now.

\section{Summarize the posteriors}\label{summarize-the-posteriors}

The model fit object contains our posterior samples, and we can use the
\texttt{joint\_summarize} function to create summaries of the
distributions.

First let's look at the \texttt{p10} parameter:

\begin{Shaded}
\begin{Highlighting}[]
\FunctionTok{joint\_summarize}\NormalTok{(goby\_fit\_cov1}\SpecialCharTok{$}\NormalTok{model, }\AttributeTok{par =} \StringTok{"p10"}\NormalTok{)}
\end{Highlighting}
\end{Shaded}

\begin{verbatim}
##      mean se_mean    sd  2.5% 97.5%    n_eff Rhat
## p10 0.003       0 0.001 0.001 0.007 15958.84    1
\end{verbatim}

Here, we see the mean, standard deviation, lower and upper bounds of the
95\% credibility interval, the effective sample size, and Rhat.

Let's plot this marginal posterior as a density plot:

\begin{Shaded}
\begin{Highlighting}[]
\FunctionTok{ggplot}\NormalTok{() }\SpecialCharTok{+}
  \FunctionTok{geom\_density}\NormalTok{(}\FunctionTok{aes}\NormalTok{(}\AttributeTok{x =} \FunctionTok{as.matrix}\NormalTok{(goby\_fit\_cov1}\SpecialCharTok{$}\NormalTok{model)[, }\StringTok{"p10"}\NormalTok{]),}
               \AttributeTok{fill =} \StringTok{"purple"}\NormalTok{, }\AttributeTok{alpha =} \FloatTok{0.4}\NormalTok{) }\SpecialCharTok{+}
  \FunctionTok{labs}\NormalTok{(}\AttributeTok{x =} \StringTok{"p10 estimate"}\NormalTok{, }\AttributeTok{y =} \StringTok{"density"}\NormalTok{) }\SpecialCharTok{+}
  \FunctionTok{theme\_minimal}\NormalTok{()}
\end{Highlighting}
\end{Shaded}

\pandocbounded{\includegraphics[keepaspectratio]{Demo1_files/figure-latex/unnamed-chunk-24-1.pdf}}

Let's plot the traceplot of \texttt{p10}:

\begin{Shaded}
\begin{Highlighting}[]
\CommentTok{\# get chains for param p10}
\NormalTok{chain1\_p10 }\OtherTok{\textless{}{-}} \FunctionTok{as.array}\NormalTok{(goby\_fit\_cov1}\SpecialCharTok{$}\NormalTok{model)[, }\StringTok{"chain:1"}\NormalTok{, }\StringTok{"p10"}\NormalTok{]}
\NormalTok{chain2\_p10 }\OtherTok{\textless{}{-}} \FunctionTok{as.array}\NormalTok{(goby\_fit\_cov1}\SpecialCharTok{$}\NormalTok{model)[, }\StringTok{"chain:2"}\NormalTok{, }\StringTok{"p10"}\NormalTok{]}
\NormalTok{chain3\_p10 }\OtherTok{\textless{}{-}} \FunctionTok{as.array}\NormalTok{(goby\_fit\_cov1}\SpecialCharTok{$}\NormalTok{model)[, }\StringTok{"chain:3"}\NormalTok{, }\StringTok{"p10"}\NormalTok{]}
\NormalTok{chain4\_p10 }\OtherTok{\textless{}{-}} \FunctionTok{as.array}\NormalTok{(goby\_fit\_cov1}\SpecialCharTok{$}\NormalTok{model)[, }\StringTok{"chain:4"}\NormalTok{, }\StringTok{"p10"}\NormalTok{]}

\FunctionTok{ggplot}\NormalTok{() }\SpecialCharTok{+}
  \FunctionTok{geom\_line}\NormalTok{(}\FunctionTok{aes}\NormalTok{(}\AttributeTok{x =} \DecValTok{1}\SpecialCharTok{:}\FunctionTok{length}\NormalTok{(chain1\_p10), }\AttributeTok{y =}\NormalTok{ chain1\_p10),}
            \AttributeTok{color =} \StringTok{"dodgerblue"}\NormalTok{) }\SpecialCharTok{+}
  \FunctionTok{geom\_line}\NormalTok{(}\FunctionTok{aes}\NormalTok{(}\AttributeTok{x =} \DecValTok{1}\SpecialCharTok{:}\FunctionTok{length}\NormalTok{(chain2\_p10), }\AttributeTok{y =}\NormalTok{ chain2\_p10),}
            \AttributeTok{color =} \StringTok{"firebrick"}\NormalTok{) }\SpecialCharTok{+}
  \FunctionTok{geom\_line}\NormalTok{(}\FunctionTok{aes}\NormalTok{(}\AttributeTok{x =} \DecValTok{1}\SpecialCharTok{:}\FunctionTok{length}\NormalTok{(chain3\_p10), }\AttributeTok{y =}\NormalTok{ chain3\_p10),}
            \AttributeTok{color =} \StringTok{"violet"}\NormalTok{) }\SpecialCharTok{+}
  \FunctionTok{geom\_line}\NormalTok{(}\FunctionTok{aes}\NormalTok{(}\AttributeTok{x =} \DecValTok{1}\SpecialCharTok{:}\FunctionTok{length}\NormalTok{(chain4\_p10), }\AttributeTok{y =}\NormalTok{ chain4\_p10),}
            \AttributeTok{color =} \StringTok{"goldenrod"}\NormalTok{) }\SpecialCharTok{+}
  \FunctionTok{labs}\NormalTok{(}\AttributeTok{x =} \StringTok{"iteration"}\NormalTok{, }\AttributeTok{y =} \StringTok{"p10 estimate"}\NormalTok{) }\SpecialCharTok{+}
  \FunctionTok{theme\_minimal}\NormalTok{()}
\end{Highlighting}
\end{Shaded}

\pandocbounded{\includegraphics[keepaspectratio]{Demo1_files/figure-latex/unnamed-chunk-25-1.pdf}}

Now let's look at the summaries of \texttt{mu}, which is the expected
catch rate at each site:

\begin{Shaded}
\begin{Highlighting}[]
\FunctionTok{joint\_summarize}\NormalTok{(goby\_fit\_cov1}\SpecialCharTok{$}\NormalTok{model, }\AttributeTok{par =} \StringTok{"mu"}\NormalTok{)}
\end{Highlighting}
\end{Shaded}

\begin{verbatim}
##           mean se_mean    sd    2.5%   97.5%    n_eff Rhat
## mu[1]    0.010   0.000 0.010   0.000   0.037 15362.94    1
## mu[2]    0.008   0.000 0.008   0.000   0.030 14686.09    1
## mu[3]    0.010   0.000 0.010   0.000   0.037 14239.15    1
## mu[4]   16.696   0.009 1.272  14.267  19.314 20010.93    1
## mu[5]    2.056   0.002 0.282   1.553   2.651 13561.08    1
## mu[6]    0.035   0.000 0.038   0.001   0.136 14026.17    1
## mu[7]    0.254   0.001 0.082   0.125   0.441 18553.03    1
## mu[8]    0.651   0.005 0.673   0.015   2.518 15246.34    1
## mu[9]    0.057   0.001 0.059   0.002   0.220 13277.94    1
## mu[10]   1.161   0.001 0.179   0.845   1.547 14443.43    1
## mu[11]  26.868   0.008 1.100  24.779  29.072 21311.11    1
## mu[12]   0.016   0.000 0.017   0.000   0.060 15268.77    1
## mu[13]   0.006   0.000 0.006   0.000   0.024 13806.36    1
## mu[14]   0.055   0.000 0.056   0.001   0.208 13980.60    1
## mu[15]   0.030   0.000 0.021   0.005   0.083 13546.27    1
## mu[16]   0.027   0.000 0.022   0.002   0.084 14365.06    1
## mu[17]   0.014   0.000 0.014   0.000   0.051 13743.58    1
## mu[18]   0.029   0.000 0.029   0.001   0.108 15724.48    1
## mu[19]   0.094   0.001 0.096   0.002   0.353 14761.51    1
## mu[20]   0.267   0.001 0.099   0.112   0.491 17966.04    1
## mu[21]   0.142   0.001 0.142   0.004   0.526 15334.63    1
## mu[22]   0.413   0.001 0.184   0.135   0.841 18768.26    1
## mu[23]  90.757   0.021 3.099  84.814  96.919 20823.27    1
## mu[24]   1.678   0.002 0.316   1.131   2.368 18928.88    1
## mu[25]  20.872   0.011 1.605  17.853  24.128 19496.94    1
## mu[26]   1.825   0.002 0.305   1.283   2.469 16213.21    1
## mu[27]   3.238   0.004 0.522   2.299   4.328 17844.60    1
## mu[28]   8.109   0.006 0.759   6.693   9.682 18146.76    1
## mu[29]   2.449   0.003 0.410   1.727   3.325 16412.54    1
## mu[30]  15.106   0.016 2.154  11.185  19.609 18508.62    1
## mu[31]  94.558   0.030 4.307  86.440 103.226 20836.56    1
## mu[32]   0.028   0.000 0.028   0.001   0.104 15975.35    1
## mu[33]   0.056   0.000 0.060   0.001   0.217 14452.03    1
## mu[34]   0.401   0.002 0.217   0.091   0.932 15809.95    1
## mu[35]   0.081   0.001 0.087   0.002   0.318 14889.43    1
## mu[36]   0.007   0.000 0.007   0.000   0.026 12668.88    1
## mu[37]   0.011   0.000 0.011   0.000   0.040 14454.33    1
## mu[38] 170.551   0.043 5.839 159.230 182.155 18422.27    1
## mu[39]   0.006   0.000 0.007   0.000   0.023 13181.25    1
\end{verbatim}

Now on to interpreting the site-level covariates:

\texttt{beta} is the parameter in the model that scales the sensitivity
of eDNA samples, relative to traditional samples, and it is site
specific and a function of our covariates:

\[
\beta_i = \alpha_1 + \alpha_2 \times FilterTime_i + \alpha_3 \times Salinity_i
\] As \texttt{beta} increases, the sensitivity of eDNA sampling
decreases. So a positive regression coefficient would indicate an
inverse relationship between the value of the covariate and the
sensitivity of eDNA.

So let's look at the marginal posterior summaries of \texttt{alpha}:

\begin{Shaded}
\begin{Highlighting}[]
\FunctionTok{joint\_summarize}\NormalTok{(goby\_fit\_cov1}\SpecialCharTok{$}\NormalTok{model, }\AttributeTok{par =} \StringTok{"alpha"}\NormalTok{)}
\end{Highlighting}
\end{Shaded}

\begin{verbatim}
##            mean se_mean    sd   2.5%  97.5%     n_eff Rhat
## alpha[1]  0.542   0.001 0.099  0.343  0.737 10313.075    1
## alpha[2]  1.022   0.001 0.121  0.777  1.252  9922.091    1
## alpha[3] -0.350   0.001 0.108 -0.560 -0.136 11829.730    1
\end{verbatim}

\texttt{alpha{[}1{]}} is the regression intercept and indicates the
sensitivity of eDNA at the \emph{average} site.

\texttt{alpha{[}2{]}} is the regression coefficient for filter time.
Since this is positive, this means that at sites where the filter time
is longer, the sensitivity of eDNA is lower.

\texttt{alpha{[}3{]}} is the regression coefficient for salinity. Since
this is negative, this means that at sites with higher salinity, the
sensitivity of eDNA is higher.

And we can go a little farther than this and use the posterior samples
and model structure to see the relationship between the value of the
covariate and of a true eDNA detection.

\begin{Shaded}
\begin{Highlighting}[]
\CommentTok{\# get posterior samples of site{-}level covariate coefficients (alpha)}
\NormalTok{samples\_alpha }\OtherTok{\textless{}{-}} \FunctionTok{as.data.frame}\NormalTok{(rstan}\SpecialCharTok{::}\FunctionTok{extract}\NormalTok{(goby\_fit\_cov1}\SpecialCharTok{$}\NormalTok{model, }
                                              \AttributeTok{pars =} \StringTok{"alpha"}\NormalTok{)}\SpecialCharTok{$}\NormalTok{alpha)}
\FunctionTok{colnames}\NormalTok{(samples\_alpha) }\OtherTok{\textless{}{-}} \FunctionTok{c}\NormalTok{(}\StringTok{"alpha1"}\NormalTok{, }\StringTok{"alpha2"}\NormalTok{, }\StringTok{"alpha3"}\NormalTok{)}
\CommentTok{\# calculate p11, given mu = 0.1 at a range of covariate values}
\NormalTok{cov\_val }\OtherTok{\textless{}{-}} \FunctionTok{seq}\NormalTok{(}\SpecialCharTok{{-}}\DecValTok{1}\NormalTok{, }\DecValTok{1}\NormalTok{, }\FloatTok{0.05}\NormalTok{)}
\NormalTok{mu }\OtherTok{\textless{}{-}} \FloatTok{0.1}
\CommentTok{\# filter time}
\NormalTok{p11\_filtertime }\OtherTok{\textless{}{-}} \FunctionTok{as.data.frame}\NormalTok{(}\FunctionTok{matrix}\NormalTok{(}\ConstantTok{NA}\NormalTok{, }\AttributeTok{nrow =} \FunctionTok{length}\NormalTok{(cov\_val), }\AttributeTok{ncol =} \DecValTok{4}\NormalTok{))}
\FunctionTok{colnames}\NormalTok{(p11\_filtertime) }\OtherTok{\textless{}{-}} \FunctionTok{c}\NormalTok{(}\StringTok{"filtertime\_z"}\NormalTok{, }\StringTok{"p11\_mean"}\NormalTok{, }\StringTok{"p11\_low"}\NormalTok{, }\StringTok{"p11\_high"}\NormalTok{)}
\ControlFlowTok{for}\NormalTok{ (i }\ControlFlowTok{in} \FunctionTok{seq\_along}\NormalTok{(}\DecValTok{1}\SpecialCharTok{:}\FunctionTok{length}\NormalTok{(cov\_val))) \{}
\NormalTok{  beta }\OtherTok{\textless{}{-}} \FunctionTok{as.matrix}\NormalTok{(samples\_alpha) }\SpecialCharTok{\%*\%} \FunctionTok{c}\NormalTok{(}\DecValTok{1}\NormalTok{, cov\_val[i], }\DecValTok{0}\NormalTok{)}
\NormalTok{  p11 }\OtherTok{\textless{}{-}}\NormalTok{ mu }\SpecialCharTok{/}\NormalTok{ (mu }\SpecialCharTok{+} \FunctionTok{exp}\NormalTok{(beta))}
\NormalTok{  p11\_filtertime[i, ] }\OtherTok{\textless{}{-}} \FunctionTok{as.vector}\NormalTok{(}\FunctionTok{c}\NormalTok{(cov\_val[i], }\FunctionTok{mean}\NormalTok{(p11),}
  \FunctionTok{ci}\NormalTok{(p11, }\AttributeTok{method =} \StringTok{"HDI"}\NormalTok{)[}\DecValTok{2}\SpecialCharTok{:}\DecValTok{3}\NormalTok{]))}
\NormalTok{\}}

\CommentTok{\# salinity}
\NormalTok{p11\_salinity }\OtherTok{\textless{}{-}} \FunctionTok{as.data.frame}\NormalTok{(}\FunctionTok{matrix}\NormalTok{(}\ConstantTok{NA}\NormalTok{, }\AttributeTok{nrow =} \FunctionTok{length}\NormalTok{(cov\_val), }\AttributeTok{ncol =} \DecValTok{4}\NormalTok{))}
\FunctionTok{colnames}\NormalTok{(p11\_salinity) }\OtherTok{\textless{}{-}} \FunctionTok{c}\NormalTok{(}\StringTok{"salinity\_z"}\NormalTok{, }\StringTok{"p11\_mean"}\NormalTok{, }\StringTok{"p11\_low"}\NormalTok{, }\StringTok{"p11\_high"}\NormalTok{)}
\ControlFlowTok{for}\NormalTok{ (i }\ControlFlowTok{in} \FunctionTok{seq\_along}\NormalTok{(}\DecValTok{1}\SpecialCharTok{:}\FunctionTok{length}\NormalTok{(cov\_val))) \{}
\NormalTok{  beta }\OtherTok{\textless{}{-}} \FunctionTok{as.matrix}\NormalTok{(samples\_alpha) }\SpecialCharTok{\%*\%} \FunctionTok{c}\NormalTok{(}\DecValTok{1}\NormalTok{, }\DecValTok{0}\NormalTok{, cov\_val[i])}
\NormalTok{  p11 }\OtherTok{\textless{}{-}}\NormalTok{ mu }\SpecialCharTok{/}\NormalTok{ (mu }\SpecialCharTok{+} \FunctionTok{exp}\NormalTok{(beta))}
\NormalTok{  p11\_salinity[i, ] }\OtherTok{\textless{}{-}} \FunctionTok{as.vector}\NormalTok{(}\FunctionTok{c}\NormalTok{(cov\_val[i], }\FunctionTok{mean}\NormalTok{(p11),}
  \FunctionTok{ci}\NormalTok{(p11, }\AttributeTok{method =} \StringTok{"HDI"}\NormalTok{)[}\DecValTok{2}\SpecialCharTok{:}\DecValTok{3}\NormalTok{]))}
\NormalTok{\}}
\CommentTok{\# plot}
\NormalTok{filtertime\_plot }\OtherTok{\textless{}{-}} \FunctionTok{ggplot}\NormalTok{(}\AttributeTok{data =}\NormalTok{ p11\_filtertime) }\SpecialCharTok{+}
  \FunctionTok{geom\_ribbon}\NormalTok{(}\FunctionTok{aes}\NormalTok{(}\AttributeTok{x =}\NormalTok{ filtertime\_z, }
                  \AttributeTok{ymin =}\NormalTok{ p11\_low, }\AttributeTok{ymax =}\NormalTok{ p11\_high), }\AttributeTok{fill =} \StringTok{"grey70"}\NormalTok{) }\SpecialCharTok{+}
  \FunctionTok{geom\_line}\NormalTok{(}\FunctionTok{aes}\NormalTok{(}\AttributeTok{x =}\NormalTok{ filtertime\_z, }\AttributeTok{y =}\NormalTok{ p11\_mean)) }\SpecialCharTok{+}
  \FunctionTok{labs}\NormalTok{(}\AttributeTok{x =} \StringTok{"filter time (normalized)"}\NormalTok{, }\AttributeTok{y =} \StringTok{"p11"}\NormalTok{) }\SpecialCharTok{+}
  \FunctionTok{theme\_minimal}\NormalTok{() }\SpecialCharTok{+}
  \FunctionTok{ggtitle}\NormalTok{(}\StringTok{"A."}\NormalTok{)}

\NormalTok{salinity\_plot }\OtherTok{\textless{}{-}} \FunctionTok{ggplot}\NormalTok{(}\AttributeTok{data =}\NormalTok{ p11\_salinity) }\SpecialCharTok{+}
  \FunctionTok{geom\_ribbon}\NormalTok{(}\FunctionTok{aes}\NormalTok{(}\AttributeTok{x =}\NormalTok{ salinity\_z,}
                  \AttributeTok{ymin =}\NormalTok{ p11\_low, }\AttributeTok{ymax =}\NormalTok{ p11\_high), }\AttributeTok{fill =} \StringTok{"grey70"}\NormalTok{) }\SpecialCharTok{+}
  \FunctionTok{geom\_line}\NormalTok{(}\FunctionTok{aes}\NormalTok{(}\AttributeTok{x =}\NormalTok{ salinity\_z, }\AttributeTok{y =}\NormalTok{ p11\_mean)) }\SpecialCharTok{+}
  \FunctionTok{labs}\NormalTok{(}\AttributeTok{x =} \StringTok{"salinity (normalized)"}\NormalTok{, }\AttributeTok{y =} \StringTok{"p11"}\NormalTok{) }\SpecialCharTok{+}
  \FunctionTok{theme\_minimal}\NormalTok{() }\SpecialCharTok{+}
  \FunctionTok{ggtitle}\NormalTok{(}\StringTok{"B."}\NormalTok{)}

\NormalTok{filtertime\_plot }\SpecialCharTok{+}\NormalTok{ salinity\_plot }\SpecialCharTok{+} \FunctionTok{plot\_layout}\NormalTok{(}\AttributeTok{nrow =} \DecValTok{1}\NormalTok{)}
\end{Highlighting}
\end{Shaded}

\pandocbounded{\includegraphics[keepaspectratio]{Demo1_files/figure-latex/unnamed-chunk-28-1.pdf}}

\section{Compare detection rates}\label{compare-detection-rates}

\section{Priors}\label{priors}

\end{document}
